%#############################################
%##                                         ##
%##   Professional CV of Drasko DRASKOVIC   ##
%##                                         ##
%#############################################

\documentclass[a4paper, oneside, final]{scrartcl}
\usepackage{soul}
\usepackage{scrpage2}
\usepackage{titlesec}
\usepackage{amsfonts,graphicx}
%\usepackage[pdfstartview=FitH,urlcolor=blue,colorlinks=true,bookmarks=true]{hyperref}
\usepackage[pdfstartview=FitH,pdfcreator={Drasko DRASKOVIC}]{hyperref}
\usepackage{marvosym}
\usepackage{tabularx,colortbl}
\pagestyle{plain}   % do page numbering ('empty' turns off)
\frenchspacing      % no aditional spaces after periods
\setlength{\parskip}{8pt}\parindent=0pt  % no paragraph indentation

\usepackage{bold-extra}

%\usepackage[T1]{fontenc}


\title{d-code}
\author{Drasko DRASKOVIC}
\begin{document}
%\date{05/13/2010}
%\maketitle
%\clearpage

\titleformat{\section}{\large\scshape\raggedright}{}{0em}{}[\titlerule]

%%%%%%%%%%%%%%%%%%
% HEADER
%%%%%%%%%%%%%%%%%% 
\begin{center}

   \textsc{\huge{\so{Drasko DRASKOVIC}}}

   % add the symbols for email and phone contact data
   \so{108 Rue Lecourbe, 75015 Paris, France}\\
   \so{{\Large\Letter} drasko.draskovic@gmail.com \ {\Large\Telefon} +33 (0)6 73487750}

\end{center}

%%%%%%%%%%%%%%%%
% OBJECTIVE
%%%%%%%%%%%%%%%%
\section{OBJECTIVE}
   A software engineer position in a results-oriented company that seeks an
   ambitious and career-conscious person, where acquired skills and education
   will be utilized toward continued growth and advancement.


%%%%%%%%%%%%%%%
% EXPERIENCE
%%%%%%%%%%%%%%%
\section{EXPERIENCE}

%%% Sequans Communications %%%
%\renewcommand{\arraystretch}{1.1}
\newcommand{\gray}{\rowcolor[gray]{.90}}
   %\begin{tabularx}{0.97\linewidth}{>{\raggedleft\scshape}p{2cm}X}
   \begin{tabularx}{1.0\linewidth}{X}
      \gray \bf\textsc{\large{Sequans Communications}} \normalfont\hfill Paris, France\\
      \gray \bf{Platform Software Engineer}\\
      \gray June 2010 --- Present (1.5 years) \\
   \end{tabularx}

\medskip

   \textit{Work on platform enablement, system and device drivers for LTE
   multi-processor SoCs, based on ARM946E-S, MIPS24Kc and Lattice Mico32
   architectures with eCos RTOS (for baseband CPU) and Linux (for application CPU)
   as a systems of choice.}

\begin{itemize}
   \item Board bring-up and HW/SW debugging (JTAG, oscilloscope,
         logic analyzer, HW modifications)
   \item Fully developed (from scratch) USIM physical layer driver and
            transport protocol layers of T=0 and T=1 protocols for Smart Card
            communication. Used Comprion and Micropross SC sniffers/analyzers
            and debugged solution. Debugged ASIC bugs and created workarounds
            in SW. Debugged HW bugs due to the rising edge timings and voltage
            levels using oscillosope and logic analyzers. 
            Designed (specification) and implemented low-level API and 
            created a suite of unitary and productions tests.
   \item Enabled OpenOCD open-source project for ARM946E-S and MIPS24Kc
            platforms, contributing patches to the community under GPL
            licence by merging separatly maintained git repositories to the
            project mainline. Wrote complex shell and TCL scripts for usage of cheap
            FTDI-based USB JTAG dongles and replacing expensive trhird-partner solutions,
            which significantly reduced development expenses for the company.
            Given a support and training to other engineers on using
            implemented solution. 
   \item Revese-engineered Lattice Mico32 MonitorROM and third-party VHDL code
            in order to develop support for OpenOCD usage of Layer 1 debugging by
            JTAG. Adapted UrJTAG for very low-level JTAG communication debugging.
   \item Fully ported uClinux to SQN3110 SoC application processor based on
            ARM946E-S.
   \item Fullty ported Linux for MIPS, OpenWRT distribution to SQN3110 FPGA
            chip, and re-wrote specific configuration, Makefile, shell and Perl scripts
            and prepared a series of patches in a build system.
   \item Fully wrote (from the scratch) scatter-gather zero-copy solution for
            inter-processor communication based on parallel communication between eCos
            and uClinux network drivers, using SW circular FIFOs and HW pointers.
            Developed Linux application to test the implementation by receiving
            RF packets sent from the eNodeB, sent via eCos driver and received
            on Linux side, and observed traffic using Wireshark.
   \item Ported SDIO Linux device drivers for SQN3110 FPGA-based SoC. Created
            workarounds (driver quirks) for ASIC bugs
            (misunderstood SDIO specification) of a third-party SDIO controller IP.
   \item Implemented series of eCos RTOS drivers (watchdog, nework activity
            GPIO, flash handling...) and low-level services. Modified
            bootloader code and NVRAM configuration in the binary form.
   \item Implemented IQ Missmatch driver for calculation phase, gain and offset
            convergence and correction, and tested solution using tone generator and
            tweaking Maxim RF chip set-up. implemented AT commands and services for
            controlling of the solution.
   \item Implemented driver for reading ISO images stored on the NOR flash and
            proper handling of sector handling and wear leveling. Implemented
            unitary test eCos application by creating ISO images on host Linux
            and turning USB dongles (dev boards) into ISO storage.
   \item Compiled and prepared toolchains for ARMEB, MIPSEB and LM32
            architectures based on Binutils v2.19, GCC v4.5.1 and GDB v7.3.
            Tested and debugged solutions on all 3 architectures and integrated
            it into development build process.
   \end{itemize}


%%% PACE %%%
\bigskip

   \begin{tabularx}{1.0\linewidth}{X}
      \gray \bf\textsc{\large{Philips (acquired by Pace)}} \normalfont\hfill Paris, France\\
      \gray \bf{Embedded Software Engineer}\\
      \gray November 2009 --- June 2010 (9 months) \\
   \end{tabularx}

\medskip

   \textit{Low level system programming for HD TV WiFi Zapper satelite set-top box solution
            based on ST7105 SoC with SH4 RISC processor,
            running embedded Linux as an OS of choice.}

\begin{itemize}
   \item Work on ethernet and USB drivers for U-Boot bootloader
   \item Bundled Linux image with intiramfs for use on 
            USB key to simulate missing flash memory
   \item Rewrote U-Boot procedures and changed low-level 
            initialization (poke table) in order to use it with 
            proprietary bootcode with strong security constraints
   \item Wrote (in assembly language) bootstrap procedure for 
            auto-decompression of system binary in specific format and 
            implemented 29/32-bit addressing mode switch needed to boot Linux image
   \item Debugged NAND Flash driver in U-Boot. Integrated similar solution in Linux system
   \item Implemented MAC address detection from NOR Flash 
            and appropriate environment setting in U-Boot and Linux
\end{itemize}


%%% SPiDCOM %%%
\bigskip

   \begin{tabularx}{1.0\linewidth}{X}
      \gray \bf\textsc{\large{SPiDCOM Technologies}} \normalfont\hfill Paris, France\\
      \gray \bf{Embedded Software Engineer}\\
      \gray December 2008 - November 2009 (1 year) \\
   \end{tabularx}

\medskip

   \textit{Firmware design and implementation for new SPiDCOM SPC300 HomePlug AV
   protocol (powerline communication) compliant SoC. Creation of BSP and
   Linux (with U-Boot) SW bundle for ARM based chip}

\begin{itemize}
   \item Implemented SDRAM controller configuration under 
            U-Boot bootloader in ARM assembly language
   \item Designed (specification creation) a custom protocol 
            for firmware update based on Ethernet-type Homeplug AV 
            MME messages and implemented it in C. Wrote PC host 
            client application (based on Linux raw sockets) to test 
            and debug protocol implementation.
   \item Fully wrote MAC controller (Ethernet) driver and enabled TFTP image boot
   \item Enabled ARM MMU (caches and pagetable set-up, 
            defined virtual to physical mapping, etc...) under U-Boot 
            in order to enable DCache, which significantly speeded up 
            Linux image transfer and all pre-boot operations
   \item Designed (specification creation) a custom protocol for 
            FW update based on Ethernet-type Homeplug AV MME messages 
            and implemented it in C. Wrote PC host client application 
            (based on Linux raw sockets) to test and debug protocol implementation
   \item Implemented (ARM assembly) code for autodetection of parameters stored on 
            flash used on board wake-up and debuged implementation with jtag and GDB. 
            Wrote OpenOCD jtag and GDB scripts to speed up debugging process
   \item Wrote (in C) Linux image boot procedure that supports dual image 
            booting with candidate selection based on parameters stored in 
            custom image headers. Wrote user space applications for creating 
            these headers and modified Buildroot Makefiles to enable their pre-pending on image build
   \item Implemented (from scratch) library for manipulating ethernet MME message 
            structures and wrote unitary and functional test for new API 
            using multi-threaded socket applications
   \item Implemented (from scratch) library for system configuration 
            and wrote unitary and functional tests
\end{itemize}


%%% Texas Instruments  %%%
\bigskip

   \begin{tabularx}{1.0\linewidth}{X}
      \gray \bf\textsc{\large{Texas Instruments}} \normalfont\hfill Nice, France\\
      \gray \bf{Multimedia Software Engineer}\\
      \gray July 2006 - December 2008 (2 year 7 months) \\
   \end{tabularx}

\medskip
   
   \textbf{DSP Architecture and Applications Group}
   \smallskip

   \textit{Module-level verification of In-Loop DeBlocking Filter and 
      Motion Estimation modules for digital video coding (compliant to several 
      modern standards: H.264, MPEG-4, etc.), part of the 
      IVA Hardware Accelerator for OMAP4 plaform} 

\begin{itemize}
   \item Fully wrote C testbenches, embedding them in the reference 
            decoder code (provided by third party vendors) and 
            programmed filter C model; Development done on Windows 
            (MS Visual C++ development environment) and Linux
   \item Ported all code to Linux and devoloped C interface 
            and synchronization mechanism to enable IPC with Specman RTL 
            simulation tool in order to test Verilog (hardware) code
   \item Wrote various Perl scripts and created Perl/Tk GUI application 
            to enable automatic running of the tests, 
            automatic logging and report creation.
   \item Defined and implemented (in C) functional coverage and algorithms 
               to extract minimum set of testng conformance video 
               bitstreams needed for full coverage
\end{itemize}


   \textbf{GSM/GPRS/EDGE L1 Software Engineer}
   \smallskip

   \textit{Layer 1 Non-Regression testing and tool development}
\begin{itemize}
   \item Provided in depth support during GSM/GPRS/EDGE 
               Layer1 Real-time embedded software development 
   \item In charge of MCU and DSP L1 SW non-regression testing 
            spread over several programs that used several 
            TI OMAP platforms with ARM 9 / ARM 11 and TI C55x DSP
   \item Extensive laboratory experience - proficiency with 
            various modern mobile telephony protocols test 
            equipment (ANITE, RACAL, CRTU, CMU200, CRTP, etc.)
   \item Analyzed L1 Trace, troobleshooted L1 sofware in case of issues
   \item Fully developed GSM Voice “audio loopback” and BER 
            test cases, implemented complete automation of 
            GSM Voice test process for RACAL AIME 6103.
   \item Developed ANITE/Agilent SW applications (in C) for testing 2G/UMA (GAN) handovers
\end{itemize}


%%%% FNX Solutions %%%
\bigskip

   \begin{tabularx}{1.0\linewidth}{X}
      \gray \bf\textsc{\large{FNX Solutions}} \normalfont\hfill Belgrade, Serbia\\
      \gray \bf{Software Engineer}\\
      \gray August 2005 - July 2006 (1 year)\\
   \end{tabularx}

\medskip

   \textit{FNX SIERRA System (system for the management and processing of
   capital market transactions) development and debugging}

\begin{itemize}
   \item Worked on system coding in C under Solaris within 
            the international team, remotely connected 
            to servers in Philadelphia, USA
   \item Designed and implemented various business objects 
            and data transfer objects used within a system for 
            database connectivity. Implemented a crucial part of 
            Oracle database port by adapting various libraries 
            used for application/database communication.
   \item Thoroughly debugged and resolved issues for risk-management 
            applications (system, network and GUI) 
   \item Programmed and designed various database triggers, 
            table and update scripts, stored procedures, etc., in SQL, 
            Transact-SQL and PL/SQL, migrating system from Sybase to Oracle
   \item Fully wrote various Perl and BASH scripts and successfully 
            implemented automatization of development process
\end{itemize}


%%% Innovational Centre of School of Electrical Engeneering %%%
\bigskip

   \begin{tabularx}{1.0\linewidth}{X}
      \gray \bf\textsc{\small{Innovational Centre of School of Electrical Engeneering}} \normalfont\hfill Belgrade, Serbia\\
      \gray \bf{Research Assistant}\\
      \gray August 2004 - August 2005 (1 year 1 month)\\
   \end{tabularx}

\medskip

   \textit{LINSEC - Linux Security and Protection System, project introducing 
            Mandatory Access Control (MAC) mechanism into Linux 
            (as opposed to existing Discretionary Access Control mechanism)}
\begin{itemize}
   \item Linux kernel hacking (file system domain access control, socket access control)
   \item Ported the legacy system on 2.6.x "vanilla" kernel and 
            resolved various errors due to GCC incompatibility issues
   \item Developed enhance for user-space tools for controlling the system patch
\end{itemize}


%%%%%%%%%%%%%%%%%%%%%%%
% OPEN SOURCE PROJECTS
%%%%%%%%%%%%%%%%%%%%%%%
\section{OPEN SOURCE PROJECTS}
GNU and FOSS evangelist, implicated in many open source projects in domains of
development, documentation or support via mailing lists.

\begin{itemize}
   \item \textbf{OpenOCD} (\url{http://openocd.sourceforge.net/})
      \begin{itemize}
         \item Added support for ARM946E-S architecture
         \item Corrected MIPS32 code and added microcodes for cache handling
         \item Documented MIPS32 implementation and functionalities
      \end{itemize}
   \item \textbf{WeIO} (\url{http://www.weio-cockpit.org})
      \begin{itemize}
         \item Project co-author and designer 
         \item Electronic components selection and ordering and soldering of
               the HW development boards using PCB re-flow technique
         \item Compiled/ported ARM CMSIS low-level libraries and drivers to NXP
                  LPC1768 (ARM CORTEX-M3) architecture
         \item Ported NuttX RTOS to WeIO platform
      \end{itemize}
   \item \textbf{Codezero OpenRISC Port} (\url{http://opencores.org/project,c0or1k})
      \begin{itemize}
         \item Main maintainer of the project 
         \item Ported C0 code OpenRISC 1000 architecture
      \end{itemize}
\end{itemize}


%%%%%%%%%%%%%%
% EDUCATION
%%%%%%%%%%%%%%
\section{EDUCATION}

   \textbf{\textsc{Diploma Engeneer (equivalent to M.Sc.)}} \\
   \textsc{School of Electrical Engineering, University of Belgrade} \\
   \textit{department of Electronics, Telecommunications and Control}

   \textbf{Master Thesis in Audiotechnics} : \textit{"Analysis of Methodes And 
         Applications for Artificial Reverberation"}, department of 
         Telecommunications, School of Electrical Engeneering, Belgrade, Serbia

\medskip
   
   Student projects:

\begin{itemize}
   \item RT application for railroad taffic signaling 
            and organization written in C++ using API of a custom RT kernel
   \item Retriggerable timer with 7-seg display 
            based on PIC16F84 uC written in assmbley (MPlab IDE)
   \item 10-band audio equilizer schematic and 
            layout implementation in Protel package
   \item JK flip-flop VLSI layout done in Magic VLSI tool
\end{itemize}


%%%%%%%%%%%%%%%%%%%%%%%
% PERSONAL INTERESTS
%%%%%%%%%%%%%%%%%%%%%%%
\section{PERSONAL INTERESTS}
\begin{itemize}
   \item Music playing (guitar) and sound and video production using GNU
            tools
   \item Contemporary and fine art photography, analogue and digital 
   \item Karate (karateka and official member of
         \href{http://www.ffkarate.fr/}{Franch Karate Federation})
   \item Film, literature and philosophy
\end{itemize}

\end{document}
